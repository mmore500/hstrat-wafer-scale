\begin{figure}

\begin{subfigure}[c]{\linewidth} \centering
\begin{minipage}[c]{0.08\linewidth} \flushright
    \caption{\rotatebox[origin=c]{90}{25 cycles}}
    \label{fig:tagged_25}
  \end{minipage}%
  \begin{minipage}[c]{0.92\linewidth}
    \includegraphics[width=\textwidth,height=0.6in,trim={0 0.81cm 0 0},clip]{binder-wse-sketches/binder/teeplots/genome=hsurftiltedsticky_tagged+replicate=e4ea2071-8228-42de-af8c-879cedff9ba7+viz=draw-biopython-tree+ext=}
  \end{minipage}%
\end{subfigure}

\vspace{-1ex}

\begin{subfigure}[c]{\linewidth} \centering
\begin{minipage}[c]{0.08\linewidth} \flushright
    \caption{\rotatebox[origin=c]{90}{50 cycles}}
    \label{fig:tagged_50}
  \end{minipage}%
  \begin{minipage}[c]{0.92\linewidth}
    \includegraphics[width=\textwidth,height=0.6in,trim={0 0.81cm 0 0},clip]{binder-wse-sketches/binder/teeplots/genome=hsurftiltedsticky_tagged+replicate=3d55af5f-7714-45da-9276-e860f46b4d94+viz=draw-biopython-tree+ext=}
  \end{minipage}%
\end{subfigure}

\vspace{-1ex}

\begin{subfigure}[c]{\linewidth} \centering
  \begin{minipage}[c]{0.08\linewidth} \flushright
    \caption{\rotatebox[origin=c]{90}{100 cycles}}
    \label{fig:tagged_100}
  \end{minipage}%
  \begin{minipage}[c]{0.92\linewidth}
    \includegraphics[width=\textwidth,height=0.6in,trim={0 0.81cm 0 0},clip]{binder-wse-sketches/binder/teeplots/genome=hsurftiltedsticky_tagged+replicate=932aa302-becb-47e8-9712-7f550b02364c+viz=draw-biopython-tree+ext=}
  \end{minipage}%
\end{subfigure}

\vspace{-1ex}

\begin{subfigure}[c]{\linewidth} \centering
  \begin{minipage}[c]{0.08\linewidth} \flushright
    \caption{\rotatebox[origin=c]{90}{250 cycles}}
    \label{fig:tagged_250}
  \end{minipage}%
  \begin{minipage}[c]{0.92\linewidth}
    \includegraphics[width=\textwidth,height=0.8in]{binder-wse-sketches/binder/teeplots/genome=hsurftiltedsticky_tagged+replicate=42dbcbb3-b803-41a4-9285-4a450bfad6ed+viz=draw-biopython-tree+ext=}
  \end{minipage}%
\end{subfigure}

\vspace{-2ex}

\caption{%
\textbf{Clade Reconstruction Trial.}
\footnotesize
Example phylogenies reconstructed from runs of increasing duration on a virtual grid of nine hardware-simulated PEs.
Founding genomes were tagged with random 16-byte identifier values, which were held constant throughout simulation (Supplementary Figure \ref{fig:genome-layout}).
Color-coding indicates each sampled taxon's founding ancestor according to this identifier value.
Simulation performed under drift conditions.
}
\label{fig:tagged}
\vspace{-0.2in}
\end{figure}
