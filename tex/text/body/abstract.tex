\begin{abstract}
Evolutionary processes underlie key questions in public health, medicine, and natural resources management.
Understanding the dynamics of these systems benefits real-world challenges like epidemiology, antibiotic resistance, cancer treatment, and conservation biology.
Phylogenetic information (i.e., the history of evolutionary relatedness among organisms) can answer important questions about the mode and tempo of evolution within these systems.
Modeling the dynamics of very large-scale evolving systems requires extensive compute power, possible using new hardware accelerators like the Cerebras Wafer Scale Engine (WSE).
However, existing phylogenetic tracking techniques for evolution simulations use a centralized, direct lineage tracking approach poorly aligned to such many-processor hardware architectures.
Recently introduced decentralized hereditary stratigraphy (hstrat) algorithms provide a potential solution for phylogenetic tracking of evolutionary simulations on the WSE, enabling analyses necessary to key science questions.
To enable phylogenetic tracking on the WSE, we developed new WSE-compatible software implementing hstrat algorithms.
Software development proceeded in three parts: (1) translating hstrat algorithms to the Cerebras Software Language, (2) validating implementations through unit tests, then (3) optimizing for distributed processing.
Then, to verify the functionality of the entire implementation, we performed end-to-end validation tests.
Our work provides means for efficient large-scale evolutionary simulations in future WSE projects, bridging key gaps between agent-based model simulations and emerging hardware accelerators.
Developed software will allow previously computationally intractable research, benefiting further explorations of evolutionary processes impacting public health, medicine, and beyond.

However, despite the strong motivation for large-scale simulation, the algorithmic pliability of evolutionary simulation, and the increasing availability of highly capable parallel, distributed, and accelerator-driven hardware platforms, significant methodological limitations currently hold back scale-up of digital evolution.
Work proposed here takes significant steps to position the agent-based evolution simulations to benefit significantly from this emerging class of hardware accelerators.

\end{abstract}
