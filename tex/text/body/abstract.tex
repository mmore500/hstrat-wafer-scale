\begin{abstract}
% Continuing improvements in computing hardware are poised to transform capabilities for \textit{in silico} modeling of cross-scale phenomena underlying major open questions in evolutionary biology and artificial life, such as transitions in individuality, eco-evolutionary dynamics, and rare evolutionary events.
Emerging ML/AI hardware accelerators, like the 850,000 processor Cerebras Wafer-Scale Engine (WSE), hold great promise to scale up the capabilities of evolutionary computation.
However, challenges remain in maintaining visibility into underlying evolutionary processes while efficiently utilizing these platforms' large processor counts.
Here, we focus on the problem of extracting phylogenetic information from digital evolution on the WSE platform.
We present a tracking-enabled asynchronous island-based genetic algorithm (GA) framework for WSE hardware.
Emulated and on-hardware GA benchmarks with a simple tracking-enabled agent model clock upwards of 1 million generations a minute for population sizes reaching 16 million.
This pace enables quadrillions of evaluations a day.
We validate phylogenetic reconstructions from these trials and demonstrate their suitability for inference of underlying evolutionary conditions.
In particular, we demonstrate extraction of clear phylometric signals that differentiate wafer-scale runs with adaptive dynamics enabled versus disabled.
Together, these benchmark and validation trials reflect strong potential for highly scalable evolutionary computation that is both efficient and observable.
Kernel code implementing the island-model GA supports drop-in customization to support any fixed-length genome content and fitness criteria, allowing it to be leveraged to advance research interests across the community.
% Developed capabilities will bring entirely new classes of previously intractable research questions within reach, benefiting further explorations within the evolutionary biology and artificial life communities across a variety of emerging high-performance computing platforms.
\end{abstract}


%% The code below is generated by the tool at http://dl.acm.org/ccs.cfm.
%% Please copy and paste the code instead of the example below.
%%
% \begin{CCSXML}
%     <ccs2012>
%        <concept>
%            <concept_id>10010147.10010257.10010293.10011809.10011812</concept_id>
%            <concept_desc>Computing methodologies~Genetic algorithms</concept_desc>
%            <concept_significance>500</concept_significance>
%            </concept>
%        <concept>
%            <concept_id>10010147.10010257.10010293.10011809.10011810</concept_id>
%            <concept_desc>Computing methodologies~Artificial life</concept_desc>
%            <concept_significance>500</concept_significance>
%            </concept>
%        <concept>
%            <concept_id>10010147.10010257.10010293.10011809.10011813</concept_id>
%            <concept_desc>Computing methodologies~Genetic programming</concept_desc>
%            <concept_significance>300</concept_significance>
%            </concept>
%        <concept>
%            <concept_id>10010147.10010341.10010349.10010362</concept_id>
%            <concept_desc>Computing methodologies~Massively parallel and high-performance simulations</concept_desc>
%            <concept_significance>500</concept_significance>
%            </concept>
%        <concept>
%            <concept_id>10010147.10010341.10010349.10010355</concept_id>
%            <concept_desc>Computing methodologies~Agent / discrete models</concept_desc>
%            <concept_significance>300</concept_significance>
%            </concept>
%      </ccs2012>
% \end{CCSXML}

% \ccsdesc[500]{Computing methodologies~Genetic algorithms}
% \ccsdesc[500]{Computing methodologies~Artificial life}
% \ccsdesc[300]{Computing methodologies~Genetic programming}
% \ccsdesc[500]{Computing methodologies~Massively parallel and high-performance simulations}
% \ccsdesc[300]{Computing methodologies~Agent / discrete models}

%%
%% Keywords. The author(s) should pick words that accurately describe
%% the work being presented. Separate the keywords with commas.
\begin{IEEEkeywords}
island-model genetic algorithm, phylogenetics, wafer-scale computing, evolutionary computation, high-performance computing, Cerebras Wafer-Scale Engine, agent-based modeling, phylogenetic tracking, evolutionary computation.
\end{IEEEkeywords}
%% A "teaser" image appears between the auth