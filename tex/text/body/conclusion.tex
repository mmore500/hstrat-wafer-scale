\section{Conclusion} \label{sec:conclusion}

Computing hardware with transformative capabilities is coming to market right now.
This fact presents an immediate opportunity to bring orders-of-magnitude greater simulation scale to bear on grand challenges in artificial life.
It is not unreasonable to anticipate possibility that with such resources some aspects of these open questions will be revealed to harbor more-is-different dispositions, in which larger scales demonstrate qualitativey different dynamics \citep{anderson1972more}.
Riding the coattails of AI-workload-driven hardware development, itself largely driven by profound more-is-different payoffs in deep learning, provides perhaps the most immediate means toward this possibility.

Such an endeavor is a community-level challenge that will require extensive collaborative effort and pursuit of outside resources.
Work presented here is an early step in methods and wide-ranging infrastructure development necessary to scale up what's possible in digital evolution research.
We have demonstrated new algorithms and software for phylogeny-enabled agent-based evolution on a next-generation HPC accelerator hardware platform.
Microbenchmarking results show that proposed instrumentation algorithms achieve several-fold improvement in computational efficiency.
Related work shows new algorithms to improve reconstruction quality in some cases, too \citep{moreno2024guide}.
Benchmarks confirm that, including tracking operations, simple models at wafer scale can achieve quintillions of replications per day.

Special characteristics set agent-based digital evolution apart from many other HPC application domains and position it as a potentially valuable testbed for innovation and leadership.
Among these factors are challenging workload heterogeneity (varying within a population and over evolutionary time), resiliency of global state to local perturbations, and perhaps unparalleled freedom to recompose underlying simulation semantics to accommodate hardware capabilities.
Indeed, artificial life and digital evolution have played host to notable boundary-pushing approaches to computing at the frontiers of computing modalities such as best-effort computing, reservoir computing, globe-spanning time-available computing, and modular tile-based computing in addition to more traditional cluster and GPU-oriented approaches \citep{moreno2021conduit,ackley2020best,ackley2023robust,heinemann2008artificial,miikkulainen2024evolving}.
Work done to scale up digital evolution simulation should be done with an eye for contributions back to broader HPC constituencies.

In this vein, presented ``surface'' indexing algorithms stand to benefit larger classes of stream curation problems, situations in which a rolling feed of sequenced observations must be dynamically downsampled to ensure retention of elements representative across observed history \citep{moreno2024algorithms}.
In particular, to further benefit observable agent-based modeling, we are interested in exploring applications that sample time-series activity at simulation sites or distill coarsened agent histories (e.g., position over time).

The immediate goal of our presented work, however, is to build means for progress toward existing research agendas held across the realms of digital evolution and artificial life.
To this end, our work has prioritized production of reusable CSL and Python software tools ready to be harnessed by the broader research community.
In particular, CSL code implementing the presented island-model GA is defined in a modularized and extensible manner to support drop-in customization to instantiate any fixed-length genome content and fitness criteria corresponding to research interests within the community.
We look forward to collaboration in broader tandem efforts to harness the Cerebras platform, and other emerging hardware, in follow-on work.
